% © Stéphane Adjemian, 2014-2017

\documentclass[10pt,slidestop]{beamer}

\usepackage{etex}

\usepackage{fourier-orns}
\usepackage{ccicons}
\usepackage{amssymb}
\usepackage{amstext}
\usepackage{amsbsy}
\usepackage{amsopn}
\usepackage{amscd}
\usepackage{amsxtra}
\usepackage{amsthm}
\usepackage{float}
\usepackage{color}
\usepackage{mathrsfs}
\usepackage{bm}
\usepackage{lastpage}
\usepackage[nice]{nicefrac}
\usepackage{setspace}
\usepackage{hyperref}
\usepackage{ragged2e}
\usepackage{listings}
\usepackage{algorithms/algorithm}
\usepackage{algorithms/algorithmic}

\usepackage{tikz,pgfplots}
\pgfplotsset{compat=newest}
\usetikzlibrary{patterns, arrows, decorations.pathreplacing, decorations.markings, calc}
\pgfplotsset{plot coordinates/math parser=false}
\newlength\figureheight
\newlength\figurewidth
\usepackage[utf8x]{inputenc}


\newcommand{\trace}{\mathrm{tr}}
\newcommand{\vect}{\mathrm{vec}}
\newcommand{\tracarg}[1]{\mathrm{tr}\left\{#1\right\}}
\newcommand{\vectarg}[1]{\mathrm{vec}\left(#1\right)}
\newcommand{\vecth}[1]{\mathrm{vech}\left(#1\right)}
\newcommand{\iid}[2]{\mathrm{iid}\left(#1,#2\right)}
\newcommand{\normal}[2]{\mathcal N\left(#1,#2\right)}
\newcommand{\dynare}{\href{http://www.dynare.org}{\color{blue}Dynare}}
\newcommand{\AllSample}{ \mathcal Y_T }
\newcommand{\sample}{\mathcal Y_T}
\newcommand{\samplet}[1]{\mathcal Y_{#1}}
\newcommand{\slidetitle}[1]{\fancyhead[L]{\textsc{#1}}}

\definecolor{gray}{gray}{0.4}

\setbeamertemplate{footline}{
{\hfill\vspace*{1pt}\href{http://creativecommons.org/licenses/by-sa/3.0/legalcode}{\ccbysa}\hspace{.1cm}
\raisebox{-.075cm}{\href{https://github.com/stepan-a/dsge-bayesian-estimation-sc}{\includegraphics[scale=.07]{../img/github.png}}}
}\hspace{1cm}}

\setbeamertemplate{navigation symbols}{}
\setbeamertemplate{blocks}[rounded][shadow=true]
\newenvironment{notes}
{\bgroup \justifying\bgroup\tiny\begin{spacing}{1.0}}
{\end{spacing}\egroup\egroup}

\begin{document}

\title{Bayesian Econometrics Primer\\ \& \\ Estimation of linearized
  DSGE models}
\author[S. Adjemian]{Stéphane Adjemian}
\institute{\texttt{stephane.adjemian@univ-lemans.fr}}
\date{June, 2017}

\begin{frame}
  \titlepage{}
\end{frame}

\section{Introduction}

\begin{frame}
  \frametitle{Introduction}
  \begin{itemize}
    \item Full information Bayesian approach to linear(ized) DSGE models.
    \item Basically, this approach allows to incorporate prior
      knowledge about the model and its parameters in the inference
      procedure.
    \item We first present the Bayesian approach in the case of a simple
      linear model for which closed form expressions are available.
    \item And we discuss issues specific to the estimation of DSGE models.
  \end{itemize}
\end{frame}


\begin{frame}
    \frametitle{Outline}
    \tableofcontents
  \end{frame}

\section{Maximum likelihood estimation}

\begin{frame}
  \frametitle{ML Estimation}
  \begin{itemize}
    \item A model ($\mathcal M$) defines a joint probability distribution parameterized
      (by $\theta_{\mathcal M}$) function over a sample of variables (say $\AllSample$):
      \begin{equation}\label{equ:bayes:model}
        f(\AllSample | \theta_{\mathcal M} , \mathcal M)
      \end{equation}
    \item The parameters $\theta_{\mathcal M}$ can be estimated by
      confronting the model to the data through:
      \begin{itemize}
      \item[--] Some moments of the DGP.
      \item[--] The probability density function of the DGP (all the moments).
      \end{itemize}
    \item The first approach is a method of moments, the second
      one is a likelihood approach.
    \item Basically, a ML estimate for $\theta_{\mathcal M}$ is
      obtained by maximizing the density of the sample with respect to
      the parameters (we seek the value of $\theta_{\mathcal M}$ that
      maximizes the ``probability of occurence'' of the sample given
      by the Nature).
    \item In the sequel, we will denote $\mathcal L(\theta) = f(\AllSample | \theta )$ the
      likelihood function, omitting the indexation with respect to the
      model when not necessary.
  \end{itemize}
\end{frame}


\begin{frame}
  \frametitle{ML Estimation}
  \framesubtitle{A simple static model}
  \begin{itemize}
    \item As a first example, we consider the following model:
      \begin{equation}
        \label{eq:simple-static-model:1}\tag{2-a}
        y_t = \mu_0 + \epsilon_t
      \end{equation}
      where $\epsilon_t\underset{\mathrm{iid}}{\sim}\normal{0}{1}$ and
      $\mu_0$ is an unknown finite real parameter.
      \bigskip
    \item According to this model, $y_t$ is normally distributed:
      \[
      y_t|\mu_0 \sim \normal{\mu_0}{1}
      \]
      and $\mathbb E[y_ty_s] = 0$ for all $s\neq t$.
      \medskip
    \item Suppose that a sample $\AllSample = \{y_1,\dots,y_T\}$ is
      available. The likelihood is defined by:
      \[
      \mathcal L (\mu) = f(y_1,\dots,y_T|\mu)
      \]
      \medskip
    \item Because the $y$s are $\mathrm{iid}$, the joint conditional
      density is equal to a product of conditional densities:
      \[
      \mathcal L (\mu) = \prod_{t=1}^T g(y_t|\mu)
      \]
    \end{itemize}
\end{frame}

\begin{frame}
  \frametitle{ML Estimation}
  \framesubtitle{A simple static model}
  \begin{itemize}
    \item Because the model is linear and Gaussian:
      \[
      \mathcal L (\mu) = \prod_{t=1}^T \frac{1}{\sqrt{2\pi}}e^{-\frac{(y_t-\mu)^2}{2}}
      \]
      \bigskip
    \item Finally we have:
      \begin{equation}
        \label{eq:simple-static-model:2}\tag{2-b}
        \mathcal L (\mu) = (2\pi)^{-\frac{T}{2}}e^{-\frac{1}{2}\sum_{t=1}^T(y_t-\mu)^2}
      \end{equation}
      \bigskip
    \item Note that the likelihood function depends on the data and the unknown parameter (otherwise we would have an identification issue).
      \bigskip
    \item Suppose that $T=1$ (only one observation in the sample). We can
      graphically determine the ML estimator of $\mu$ in this case.
    \end{itemize}
\end{frame}

\begin{frame}
  \frametitle{ML Estimation}
  \framesubtitle{A simple static model (cont'd)}
  \begin{center}
    \begin{tikzpicture}
      \begin{axis}[
  no markers,
  domain=0:6,
  ymin=0,
  axis lines=left,
  xlabel=$\AllSample$,
  ylabel=$\mathcal L(\mu)$,
  every axis y label/.style={at=(current axis.above origin),anchor=south},
  every axis x label/.style={at=(current axis.right of origin),anchor=west},
  height=6cm,
  width=10cm,
  xtick=\empty,
  ytick=\empty,
  enlargelimits=false,
  clip=false,
  axis on top,
  grid = major,
  ]
 \addplot [thick, samples=400] {1/(.5*sqrt(2*pi))*exp(-((x-1)^2)/(2*.5^2))} node[pos=0.345] {$\bullet$} ;
 \addplot [thick, samples=400, red] {1/(.5*sqrt(2*pi))*exp(-((x-1.8)^2)/(2*.5^2))} node[pos=1.8] {$\bullet$};
 \addplot [dashed, gray] coordinates {(1.8,0) (1.8, .81)} node[pos=0,below] {\small $y_1$} ;
 \addplot [dashed, gray] coordinates {(0, .225) (1.8, .225)} node[pos=0,left] {\small $f(y_1|\mu=\bar\mu)$};
\end{axis}

% Local Variables:
% TeX-master: "../tex/slides.tex"
% ispell-check-comments: exclusive
% ispell-local-dictionary: "american-insane"
% End:
    \end{tikzpicture}
  \end{center}
  {\small Clearly, the value of the density of $y_1$ conditional on
    $\mu$, \emph{ie} the likelihood, is maximized for $\mu=y_1$: for
  any $\bar\mu\neq y_1$ we have $f(y_1|\mu=\bar\mu)<f(y_1|\mu=y_1)$}
\end{frame}


\begin{frame}
  \frametitle{ML Estimation}
  \framesubtitle{A simple static model (cont'd)}
  \begin{itemize}
  \item[$\Rightarrow$] If we have only one observation, $y_1$, the
    Maximum Likelihood estimator is equal to the observation:
    $\widehat{\mu} = y_1$.
    \bigskip
    \item This estimator is unbiased and its variance is 1.
      \bigskip
    \item More generally, one can show that the maximum likelihood
      estimator is equal to the sample mean:
      \begin{equation}
        \label{eq:simple-static-model:2}\tag{2-c}
        \widehat{\mu}_{T} = \frac{1}{T}\sum_{t=1}^Ty_t
      \end{equation}
    \item This estimator is unbiased and its variance is given by:
      \begin{equation}
        \label{eq:simple-static-model:2}\tag{2-d}
        \mathbb V\left[\widehat{\mu}_{T}\right] = \frac{1}{T}
      \end{equation}
      \item Because $\mathbb V\left[\widehat{\mu}\right]$ goes to
        zero as the sample size goes to infinity, we know that this
        estimator converges in probability to the true value $\mu_0$
        of the unknown parameter:
        \[
        \widehat{\mu}_{T}\overset{\text{\tiny{proba}}}{\underset{T\rightarrow\infty}{\longrightarrow}} \mu_0
        \]
  \end{itemize}
\end{frame}

\begin{frame}
  \frametitle{ML Estimation}
  \framesubtitle{A simple dynamic model}
  \begin{itemize}
  \item Suppose that the data are generated by an AR(1) model:
    \[
    y_t = \varphi y_{t-1} + \epsilon_t
    \]
    with $|\varphi|<1$ and
    $\epsilon_t\underset{\mathrm{iid}}{\sim}\normal{0}{\sigma_{\epsilon}^2}$.

    \medskip

  \item In this case, $y_t$ depends (directly) on  $y_{t-1}$ and also
    on $y_{t-2}, y_{t-3}, ...$.

    \medskip

  \item It is no more legal to write the likelihood as the  as a product of
    marginal densities of the observations.

    \medskip

  \item The joint density of $\mathrm y \equiv
      (y_t,y_{t+1},\dots,y_{t+H-1})'$ is given by:
      \[
      f(\mathrm y) = (2\pi)^{-\frac{H}{2}}|\Sigma_y|^{-\frac{1}{2}}e^{-\frac{1}{2}\mathrm
        y'\Sigma_{y}^{-1}\mathrm y}
      \]
      with
      {\tiny\[
      \Sigma_y = \frac{\sigma^{2}_{\epsilon}}{1-\varphi^2}
      \begin{pmatrix}
        1    & \varphi         & \varphi^2 & \dots     & \dots   & \varphi^{H-1} \\
  \varphi    & 1               & \varphi   & \varphi^2 & \dots   & \varphi^{H- 2} \\
   \vdots    &                 &           &           &         &               \\
\varphi^{H-1} & \varphi^{H- 2}   & \dots      & \dots    & \varphi  & 1
      \end{pmatrix}
      \]}
    under the assumption of stationarity $\Rightarrow$ Likelihood function.
  \end{itemize}
  \end{frame}


\begin{frame}
  \frametitle{ML Estimation}
  \framesubtitle{A simple dynamic model (cont'd)}

    \begin{itemize}
      \item The inverse of the covariance
      matrix, $\Sigma_y$, can be factorized as $\Sigma_y ^{-1}=
      \sigma_{\epsilon}^{-2}L'L$ with:
      {\tiny\[
        L =
        \begin{pmatrix}
          \sqrt{1-\varphi^2} & 0 & 0 & \dots & 0 & 0\\
          -\varphi & 1 & 0 &\dots & 0 & 0 \\
          0 & -\varphi & 1 & \dots & 0 & 0 \\
          \vdots & &&&& \vdots \\
          0 & &&&-\varphi & 1
        \end{pmatrix}
      \]}
    a $T\times T$ matrix.

\bigskip

    \item The likelihood function can be
    written equivalently as:\newline
    \[
      \begin{split}
        \mathcal L(\varphi,\sigma_{\epsilon}^2) =
    {\color{blue}(2\pi)^{-\frac{T-1}{2}}}&{\color{blue}\sigma_{\epsilon}^{T-1}e^{-\frac{1}{2\sigma_{\epsilon}^2}\sum_{t=2}^T\left(y_t-\varphi
      y_{t-1}\right)^2}}\\
    \times {\color{red}(2\pi)^{-\frac{1}{2}}}&{\color{red}\left(\frac{\sigma_{\epsilon}^2}{1-\varphi^2}\right)^{-\frac{1}{2}}
    e^{-\frac{1-\varphi^2}{2\sigma_{\epsilon}^2}y_1^2}}
      \end{split}
    \]

    \bigskip

    {\color{blue} --- Product of conditional densities,
      $y_t|y_{\underline{t-1}}$}\newline

    {\color{red} --- Marginal density of the initial condition, $y_1$}\newline
    \end{itemize}
  \end{frame}

\section{Prior and posterior beliefs}

\begin{frame}
  \frametitle{Bayes theorem}

  \begin{itemize}

  \item Let $A$ and $B$ be two events.

    \bigskip

  \item Let $\mathbb P(A)$ and  $\mathbb P(B)$ be the marginal
    probabilities of these events.

    \bigskip

  \item Let $\mathbb P(A \cap B)$ be the joint probability of events $A$
    and $B$.

    \bigskip

  \item The Bayes theorem states that the probability of $B$
    conditional on $A$ is given by:
    \[
    \mathbb P(B|A) = \frac{\mathbb P(A \cap B)}{\mathbb P(A)}
    \]

    \bigskip

  \item Or equivalently, that a joint probability can be expressed as
    the product of a conditional density and a marginal density:
    \[
      \mathbb P(A \cap B) = \mathbb P(B|A)\mathbb P(A)\quad\text{ or }\quad \mathbb P(A \cap B) = \mathbb P(A|B)\mathbb P(B)
    \]

  \bigskip

  \item Conditioning inversion:
    \[
      \Rightarrow \mathbb P(B|A) = \frac{\mathbb P(A|B)\mathbb P(B)}{\mathbb P(A)}
    \]

  \bigskip

  \item Same for continuous random variables.

  \end{itemize}

   \end{frame}


\begin{frame}
  \frametitle{Prior and posterior beliefs}

  \begin{itemize}

  \item We assume that we are able to characterize our prior knowledge
    about a parameter with a probability density function.

  \item Let $p_0(\theta)$ be the prior density characterizing our beliefs about the vector
    of parameters $\theta$.

  \item Our aim is to update our (prior) beliefs about $\theta$ with
    the sample information ($\AllSample$) embodied in the likelihood
    function, $\mathcal L(\theta) = f(\AllSample|\theta)$.

  \item We define the posterior density, $p_1(\theta|\AllSample)$,
    which represents our updated beliefs.

   \item By the Bayes theorem we have:
     \[
     p(\theta|\AllSample) = \frac{g(\theta,\AllSample)}{p(\AllSample)}
     \]
     and
     \[
      p(\theta|\AllSample) = \frac{f(\theta|\AllSample)p_0(\theta)}{p(\AllSample)}
     \]
     where $g$ is the joint density of the sample and the parameters.
  \end{itemize}

   \end{frame}


\begin{frame}
  \frametitle{Prior and posterior beliefs (cont'd)}

  \begin{itemize}

  \item The posterior density is given by:
    \[
    p(\theta|\AllSample) = \frac{\mathcal L(\theta)p_0(\theta)}{p(\AllSample)}
    \]

  \item Noting that the denominator does not depend on the parameters,
    we have that the posterior density is proportional (w.r.t
    $\theta$) to the product of the likelihood and the prior density:
    \[
    p(\theta|\AllSample) \propto \mathcal L(\theta)p_0(\theta)
    \]

  \item All the posterior inference about the
    parameters can be done with the posterior kernel: $\mathcal
    L(\theta)p_0(\theta)$.

  \item The denominator is the marginal density of the sample. Because
    a density has to sum up to one, we have:
    \[
    p(\AllSample) = \int f(\AllSample|\theta)p_0(\theta)\mathrm d\theta
    \]
    The marginal density is a weighted average of the likelihood
    function $\rightarrow$ will be used later for model comparison.
  \end{itemize}

\end{frame}


\begin{frame}
  \frametitle{Prior and posterior beliefs}
  \framesubtitle{A simple static model (cont'd)}
  \begin{itemize}
  \item For the sake of simplicity, we choose a
    Gaussian prior for the parameter $\mu$, with prior expectation
    $\mu_0$ and prior variance $\sigma_{\mu}^2$:
    \[
    p_0(\mu) = \frac{1}{\sigma_{\mu}\sqrt{2\pi}}e^{-\frac{1}{2\sigma_{\mu}^2}\left(\mu-\mu_0\right)^2}
    \]
  \item The smaller is the prior variance, $\sigma_{\mu}^2$, the more
    informative is the prior.
  \item The posterior density is proportional to the product of the
    prior density and the likelihood:
    \[
    p_1(\mu|\AllSample) \propto \frac{1}{\sigma_{\mu}\sqrt{2\pi}}e^{-\frac{1}{2\sigma_{\mu}^2}\left(\mu-\mu_0\right)^2}(2\pi)^{-\frac{T}{2}}e^{-\frac{1}{2}\sum_{t=1}^T(y_t-\mu)^2}
    \]
  \item One can show that the righthand side expression is
    proportional to a Gaussian density.
  \end{itemize}
\end{frame}

\begin{frame}
  \frametitle{Prior and posterior beliefs}
  \framesubtitle{A simple static model (cont'd)}

  \begin{itemize}

  \item The likelihood can be equivalently written as:
    \[
    \mathcal L(\mu) =  (2\pi)^{-\frac{T}{2}}e^{-\frac{1}{2}\left(\nu s^2 + T \left(\mu-\widehat{\mu}\right)^2\right)}
    \]
    with $\nu=T-1$ and
    \[
    s^2 = \frac{1}{\nu}\sum_{t=1}^T\left(y_t-\widehat{\mu}\right)^2
    \]

  \item $s^2$ and $\widehat{\mu}$ are sufficient statistics:
    they convey all the necessary sample information regarding the
    inference w.r.t $\mu$.

  \item We use this alternative expression of the likelihood to show
    that the posterior density is Gaussian. We have:
    \[
    p_1(\mu|\AllSample) \propto \frac{1}{\sigma_{\mu}\left(\sqrt{2\pi}\right)^{T+1}}e^{-\frac{1}{2\sigma_{\mu}^2}\left(\mu-\mu_0\right)^2-\frac{\nu}{2}s^2 - \frac{T}{2} \left(\mu-\widehat{\mu}\right)^2}
    \]

  \end{itemize}
\end{frame}


\begin{frame}
  \frametitle{Prior and posterior beliefs}
  \framesubtitle{A simple static model (cont'd)}

  \begin{itemize}

  \item We can simplify the previous expression by omitting all the
    multiplicative terms not related to $\mu$ (this is legal because we are
    interested in a proportionality w.r.t $\mu$):
    \[
    p_1(\mu|\AllSample) \propto e^{-\frac{1}{2\sigma_{\mu}^2}\left(\mu-\mu_0\right)^2 - \frac{T}{2} \left(\mu-\widehat{\mu}\right)^2}
    \]

  \item We develop the quadratic forms and remove all the terms
    appearing additively (under the exponential function); we obtain:
    \[
    p_1(\mu|\AllSample) \propto
    e^{-\frac{1}{2}\left(\sigma_{\mu}^{-2}+T\right)\left(\mu
      - \frac{T\widehat{\mu}+\mu_0\sigma_{\mu}^{-2}}{T+\sigma_{\mu}^{-2}}\right)^2}
    \]

  \item We recognize the expression of a Gaussian density (up to a
    scale parameter that does not depend on $\mu$).
  \end{itemize}
\end{frame}

\begin{notes}
  Let $A(\mu) = \frac{1}{\sigma_{\mu}^2}\left(\mu-\mu_0\right)^2 +
  T\left(\mu-\widehat{\mu}\right)^2$. We establish the
  last expression of the posterior kernel by rewriting $A(\mu)$ as:
  \[
\begin{split}
A(\mu) &=  T(\mu-\widehat{\mu})^2
        +\frac{1}{\sigma_{\mu}^2}(\mu-\mu_0)^2\\
 &= T\left(\mu^2+\widehat{\mu}^2-2\mu\widehat{\mu}\right)
+\frac{1}{\sigma_{\mu}^2}\left(\mu^2+\mu_0^2-2\mu\mu_0\right)\\
&= \left(T+\frac{1}{\sigma_{\mu}^2}\right)\mu^2
-2\mu\left(T\widehat{\mu}+\frac{1}{\sigma_{\mu}^2}\mu_0\right)
+\left(T\widehat{\mu}^2+\frac{1}{\sigma_{\mu}^2}\mu_0^2\right)\\
&=\left(T+\frac{1}{\sigma_{\mu}^2}\right)
\left[
\mu^2
-2\mu\frac{T\widehat{\mu}+\frac{1}{\sigma_{\mu}^2}\mu_0}
{T+\frac{1}{\sigma_{\mu}^2}}
\right]
+\left(T\widehat{\mu}^2+\frac{1}{\sigma_{\mu}^2}\mu_0^2\right)\\
&= {\color{red}\left(T+\frac{1}{\sigma_{\mu}^2}\right)
\left[
\mu-
\frac{T\widehat{\mu}+\frac{1}{\sigma_{\mu}^2}\mu_0}
{T+\frac{1}{\sigma_{\mu}^2}}
\right]^2}
+\left(T\widehat{\mu}^2+\frac{1}{\sigma_{\mu}^2}\mu_0^2\right)\\
&\quad\quad\quad\quad\quad\quad\quad\quad\quad\quad\quad\quad\quad\quad\quad\quad-\frac{
\left(T \widehat{\mu} +\frac{1}{ \sigma_{\mu }^2} \mu_0\right)^2 }
{T+\frac{1}{\sigma_{\mu}^2}}
\end{split}
\]
In the last equality, the two last additive terms do not depend on
$\mu$ and can be therefore omitted.
\end{notes}


\begin{frame}
  \frametitle{Prior and posterior beliefs}
  \framesubtitle{A simple static model (cont'd)}

  \begin{itemize}
  \item The posterior distribution is Gaussian with (posterior) expectation:
    \[
    \mathbb E \left[\mu\right] =
    \frac{T\widehat{\mu}+\frac{1}{\sigma_{\mu}^2}\mu_0}
    {T+\frac{1}{\sigma_{\mu}^2}}
    \]
    and (posterior) variance:
    \[
    \mathbb V \left[\mu\right] = \frac{1}
{T+\frac{1}{\sigma_{\mu}^2}}
    \]
  \item As soon as the amount of prior information is positive
    ($\sigma_{\mu}^2<\infty$) the posterior variance is less than the
    variance of the maximum likelihood estimator ($\nicefrac{1}{T}$).

\bigskip

  \item The posterior expectation is a convex combination of the
    maximum likelihood estimator and the prior expectation.
  \end{itemize}
\end{frame}


\begin{frame}
  \frametitle{Prior and posterior beliefs}
  \framesubtitle{A simple static model (cont'd)}

  \begin{itemize}
  \item The Bayesian approach can be interpreted as a bridge between
    the calibration approach ($\sigma_{\mu}^2=0$, infinite amount of
    prior information) and the ML approach
    ($\sigma_{\mu}^2=\infty$, no prior information):
    \[
    \mathbb E \left[\mu\right] \xrightarrow[\sigma_{\mu}^2\rightarrow 0]{}\mu_0
    \]
    and
    \[
    \mathbb E \left[\mu\right] \xrightarrow[\sigma_{\mu}^2\rightarrow \infty]{}\widehat{\mu}
    \]

    \bigskip

  \item The more important is the amount of information in the sample,
    the smaller will be the gap between the posterior expectation and
    the ML estimator.
  \end{itemize}
\end{frame}

\section{Point estimate}

\begin{frame}
  \frametitle{Point estimate}

  \begin{itemize}
  \item The outcome of the Bayesian approach is a (posterior)
    probability density function.
  \item But people generally expect much less information: a point
    estimate is often enough for most practical purposes (a single
    value for each parameter with a measure of uncertainty).
  \item[$\Rightarrow$] We need to reduce a distribution to a
    ``representative'' point.
\bigskip
  \item Let $L(\theta,\widehat{\theta})$ be the loss incurred if we
    choose $\widehat{\theta}$ while $\theta$ is the true value.
\bigskip
  \item The idea is to choose the value of $\theta$ that minimizes
    this loss... But the true value of $\theta$ is obviously unknown,
    so we minimize the (posterior) expected loss instead:
      \[
      \theta^{\star} = \arg\min_{\widehat{\theta}} \mathbb E
      \left[L(\theta,\widehat{\theta})\right] =
      \arg\min_{\widehat{\theta}} \int
      L(\theta,\widehat{\theta})p_1(\theta|\AllSample)\mathrm d\theta
      \]
  \item The choice of the loss function is purely arbitrary, for each
    loss we will obtain a different point estimate.
  \end{itemize}
\end{frame}


\begin{frame}
  \frametitle{Point estimate}
  \framesubtitle{Quadratic loss function ($L_2$ norm)}
  \begin{itemize}
  \item Suppose that the loss function is quadratic:
    \[
    L(\theta,\widehat{\theta}) = (\theta-\widehat{\theta})'\Omega(\theta-\widehat{\theta})
    \]
    where $\Omega$ is a symmetric positive definite matrix. Note that
    this function returns a (real) scalar.
    \bigskip
  \item The (posterior) expectation of the loss is:
    \medskip
    {\tiny\[
    \begin{split}
      \mathbb  E \left[L(\theta,\widehat{\theta})\right] &=
      \mathbb  E \left[(\theta-\widehat{\theta})'\Omega(\theta-\widehat{\theta})\right]\\
      &= \mathbb  E \left[\left(\theta-\mathbb E\theta-\left(\widehat{\theta}-\mathbb E\theta\right)\right)'\Omega\left(\theta-\mathbb E\theta-\left(\widehat{\theta}-\mathbb E\theta\right)\right)\right]\\
      &= \mathbb  E \left[\left(\theta-\mathbb E\theta\right)'\Omega\left(\theta-\mathbb E\theta\right)\right]+(\widehat{\theta}-\mathbb E\theta)'\Omega(\widehat{\theta}-\mathbb E\theta)
    \end{split}
    \]}
  \bigskip
  \item Noting that the first term does not depend on the choice
    variable, $\widehat{\theta}$, the expected loss is trivially
    minimized when $\widehat{\theta}$ is equal to the (posterior)
    expectation of $\theta$:
    \[
    \theta^{\star} = \mathbb E \left[\theta\right]
    \]
    \item[$\Rightarrow$] If the loss is quadratic the optimal point
      estimate is the posterior expectation.
  \end{itemize}
\end{frame}


\section{Marginal density of the sample}

\begin{frame}
  \frametitle{Marginal density of the sample}
  \begin{itemize}
  \item If we are only interested in inference about the parameters,
    the  marginal density of the data, $p(\AllSample)$, can be omitted.
  \item We already saw that the marginal density of the data is:
    \[
    p(\AllSample) = \int f(\AllSample|\theta)p_0(\theta)\mathrm d\theta
    \]
    \begin{itemize}
    \item The marginal density of the sample acts as a constant of
      integration in the expression of the posterior density.
    \item The marginal density of the sample is an average of the likelihood
      function (for different values of the estimated parameters)
      weighted by the prior density.
    \end{itemize}
    \bigskip
  \item[$\Rightarrow$] The marginal density of the sample is a measure
    of fit, which does not depend on the parameters (because we
    integrate them out).
    \bigskip
  \item Note that, theoretically, it is possible to compute the
    marginal density of the sample (conditional on a model) without
    estimating the parameters.
  \end{itemize}
\end{frame}


\begin{frame}
  \frametitle{Marginal density of the sample}
  \framesubtitle{A simple static model (cont'd)}
  \begin{itemize}
  \item Suppose again that the sample size is $T=1$. The likelihood is
    given by:
    \[
    f(\AllSample|\mu) = \frac{1}{\sqrt{2\pi}}e^{-\frac{1}{2}(y_1-\mu)^2}
    \]
  \item The marginal density is then given by:
    \[
    \begin{split}
      p(\AllSample) &=
      \int_{-\infty}^{\infty}f(y_1|\mu)p_0(\mu)\mathrm d\mu \\
      &=
      (2\pi\sigma_{\mu})^{-1}\int_{-\infty}^{\infty}e^{-\frac{1}{2}\left((y_1-\mu)^2+\frac{(\mu-\mu_0)^2}{\sigma_{\mu}^2}\right)}\mathrm
      d\mu\\
      &=\frac{1}{\sqrt{2\pi(1+\sigma_{\mu}^2)}}e^{-\frac{(y_1-\mu_0)^2}{2(1+\sigma_{\mu}^2)}}
    \end{split}
    \]
    \item We can directly obtain the same result by noting that $y_1$
      is the sum of two Gaussian random variables: $\normal{0}{1}$ and $\normal{\mu_0}{\sigma_{\mu}^2}$.
  \end{itemize}
\end{frame}


\begin{frame}
  \frametitle{Marginal density of the sample}
  \framesubtitle{Model comparison}
  \begin{itemize}

  \item Suppose we have two models $\mathcal{A}$ and $\mathcal{B}$ (with two associated
    vectors of deep parameters $\theta_\mathcal{A}$ and $\theta_\mathcal{B}$) estimated
    using \emph{the same sample} $\AllSample$.
    \bigskip
  \item For each model $\mathcal{I}=\mathcal{A},\mathcal{B}$ we can evaluate, at least
    theoretically, the marginal density of the data conditional on the model:
    \[
    p(\AllSample|\mathcal{I}) = \int
    f(\sample |\theta_{\mathcal{I}},\mathcal{I})p_0(\theta_{\mathcal{I}}|\mathcal{I}) d\theta_{\mathcal{I}}
    \]
    by integrating out the deep parameters $\theta_{\mathcal{I}}$ from
    the posterior kernel.
    \bigskip
  \item $p(\sample|\mathcal{I})$ measures the fit of model
    $\mathcal{I}$. If we have to choose between models $\mathcal A$
    and $\mathcal B$ we will select the model with the highest marginal
    density of the sample.
    \bigskip
  \item Note that models $\mathcal A$ and $\mathcal B$ need not to be
    nested (for instance, we do not require that $\theta_{\mathcal A}$
    be a subset of $\theta_{\mathcal B}$) for the comparison to make
    sense, because the compared marginal densities do not
    depend on the parameters. The classical approach
    (comparisons of likelihoods) by requiring nested models is much
    less obvious.
  \end{itemize}
\end{frame}


\begin{frame}
  \frametitle{Marginal density of the sample}
  \framesubtitle{Model comparison (cont'd)}
  \begin{itemize}

        \item Suppose we have a prior distribution over models $\mathcal A$ and $\mathcal B$:
          $p(\mathcal{A})$ and $p(\mathcal{B})$.
          \bigskip
        \item Again, using the Bayes theorem we can compute the posterior
            distribution over models:
            \[
            p(\mathcal{I}|\sample) = \frac{p(\mathcal{I})p(\sample|\mathcal{I})}
            {\sum_{\mathcal{I}=\mathcal{A},\mathcal{B}}p(\mathcal{I})p(\sample|\mathcal{I})}
            \]

        \item This formula may easily be generalized to a collection of $N$ models.
          \bigskip

        \item In the literature posterior odds ratio, defined as:
          \[
          \frac{p(\mathcal{A}|\sample)}{p(\mathcal{B}|\sample)} =
          \frac{p(\mathcal{A})}{p(\mathcal{B})}
          \frac{p(\sample|\mathcal{A})}{p(\sample|\mathcal{B})}
          \]
          are often used to discriminate between different models. If
          the posterior odds ratio is large ($>$100) we can safely
          choose model $\mathcal A$.

  \end{itemize}
\end{frame}


\begin{frame}
  \frametitle{Marginal density of the sample}
  \framesubtitle{Model comparison (cont'd)}
  \begin{itemize}
    \item Note that we do not necessarily have to choose one model.
      \bigskip
    \item Even if a model has a smaller posterior probability (or
      marginal density) it may provide useful informations in some
      directions (or frequencies), so we should not discard this
      information.
      \bigskip
    \item An alternative is to mix the models.
      \bigskip
    \item If these models are used for forecasting inflation, we can
      report an average of the forecasts weighted by the posterior
      probabilities, $p(\mathcal{I}|\sample)$, instead of the
      forecasts of the best model (in terms of marginal density)
      $\rightarrow$ Bayesian averaging.
  \end{itemize}
\end{frame}

\section{Forecasts}

\begin{frame}
  \frametitle{Predictive density}
  \begin{itemize}
  \item We often seek to use the estimated model to do inference about
    unobserved variables.
  \item The most obvious example is the forecasting exercise.
  \item In the Bayesian context the density of an unobserved variable
    (for instance the future growth of GDP) given the sample, is
    called a predictive density.
  \item Let $\tilde{y}$ be a vector of unobserved variables. The joint
    posterior density of $\tilde{y}$ and $\theta$ is:
    \[
    p_1(\tilde{y},\theta|\AllSample) =g(\tilde{y}|\theta,\AllSample)p_1(\theta|\AllSample)
    \]
  \item The posterior predictive density is obtained by integrating out the
    parameters:
    \[
    p(\tilde{y}|\AllSample) = \int
    g(\tilde{y}|\theta,\AllSample)p_1(\theta|\AllSample)\mathrm d\theta
    \]
    The posterior predictive density is the average of the density of
    $\tilde{y}$ knowing the parameters weighted by the posterior
    density of the parameters.
  \end{itemize}
\end{frame}


\begin{frame}
  \frametitle{Predictive density}
  \framesubtitle{A simple static model (cont'd)}
  \begin{itemize}
  \item Suppose that we want to do inference about the out of sample
    variable $y_{T+1}$ (forecast).
    \bigskip
  \item The density of $y_{T+1}$ conditional on the sample and on the
    parameter is:
    \[
    g(y_{T+1}|\mu,\AllSample) \propto e^{-\frac{1}{2}(y_{T+1}-\mu)^2}
    \]
    Note that this conditional density does not depend on $\AllSample$ because the
    model is static (for an autoregressive model, at least $y_T$
    would appear under the quadratic term).
    \bigskip
  \item Remember that the posterior density of $\mu$ is:
    \[
    p_1(\mu|\AllSample) \propto e^{-\frac{1}{2\mathbb
        V[\mu]}(\mu-\mathbb E[\mu])^2}
    \]
    where $\mathbb E[\mu]$ and $\mathbb V[\mu]$ are the posterior first
    and second order moments obtained earlier.
  \end{itemize}
\end{frame}


\begin{frame}
  \frametitle{Predictive density}
  \framesubtitle{A simple static model (cont'd)}
  \begin{itemize}
  \item The predictive density for $y_{T+1}$ is given by:
    \[
    \begin{split}
      p(y_{T+1}|\AllSample) &= \int
    g(y_{T+1}|\mu,\AllSample)p_1(\mu|\AllSample)\mathrm d\mu\\
    &\propto \int_{-\infty}^{\infty}e^{-\frac{1}{2}(y_{T+1}-\mu)^2-\frac{1}{2\mathbb
        V[\mu]}(\mu-\mathbb E[\mu])^2}\mathrm d\mu
    \end{split}
    \]
    \item The terms under the exponential in the
    last expression can be rewritten as:
    \[
    \left[1+\frac{1}{\mathbb
        V[\mu]}\right]\left(\mu-\frac{y_{T+1}+\mathbb
        V[\mu]^{-1}}{1+\mathbb V[\mu]^{-1}}\right)^2+\frac{\mathbb
      V[\mu]^{-1}}{1+\mathbb V[\mu]^{-1}}\left(y_{T+1}-\mathbb E[\mu]\right)^2
    \]
  \end{itemize}
\end{frame}


\begin{frame}
  \frametitle{Predictive density}
  \framesubtitle{A simple static model (cont'd)}
  \begin{itemize}
  \item By substitution in the expression of the predictive density for $y_{T+1}$ we obtain:
    \[
    \begin{split}
       p(y_{T+1}|\AllSample) &\propto
       \int_{-\infty}^{\infty}e^{-\frac{1}{2}\left[1+\frac{1}{\mathbb
        V[\mu]}\right]\left(\mu-\frac{y_{T+1}+\mathbb
        V[\mu]^{-1}}{1+\mathbb V[\mu]^{-1}}\right)^2
  -\frac{1}{2}\frac{\mathbb
      V[\mu]^{-1}}{1+\mathbb V[\mu]^{-1}}\left(y_{T+1}-\mathbb
      E[\mu]\right)^2}\mathrm d\mu\\
  &\propto e^{-\frac{1}{2}\frac{\mathbb
      V[\mu]^{-1}}{1+\mathbb V[\mu]^{-1}}\left(y_{T+1}-\mathbb
      E[\mu]\right)^2}\int_{-\infty}^{\infty}e^{-\frac{1}{2}\left[1+\frac{1}{\mathbb
        V[\mu]}\right]\left(\mu-\frac{y_{T+1}+\mathbb
        V[\mu]^{-1}}{1+\mathbb V[\mu]^{-1}}\right)^2}\mathrm d\mu\\
  &\propto e^{-\frac{1}{2}\frac{1}{1+\mathbb V[\mu]}\left(y_{T+1}-\mathbb
      E[\mu]\right)^2}
    \end{split}
    \]
  \item Unsurprisingly, we recognize the Gaussian density:
    \[
    y_{T+1}|\AllSample \sim \normal{\mathbb E[\mu]}{1+\mathbb V[\mu]}
    \]
  \item We would have obtained directly the same result by first
    noting that $y_{T+1}$ is the sum of two Gaussian random variables:
    $\normal{\mathbb E[\mu]}{\mathbb V[\mu]}$ (for the estimated
    parameter) and $\normal{0}{1}$ (for the error term).
  \end{itemize}

\end{frame}


\begin{frame}
  \frametitle{Predictive density}
  \framesubtitle{Point prediction}
  \begin{itemize}
  \item For reporting our forecast, we may want to select one point in
    the predictive distribution.
    \bigskip
  \item We proceed as for the point estimate by choosing an arbitrary
    loss function and minimizing the posterior expected loss.
    \bigskip
  \item Usually the expectation of the predictive distribution is
    reported (rationalized with a quadratic loss function).
  \end{itemize}

\end{frame}


\section{Likelihood of linear DSGE models}


\begin{frame}
  \frametitle{Estimation of DSGE models}
  \framesubtitle{Structural form}
  \begin{itemize}
  \item We suppose that the DSGE model can be cast in the following form:
  \begin{equation}\label{equ:dsge:def}
    \mathbb E_t \left[\mathcal
      F_{\theta}(y_{t+1},y_{t},y_{t-1},\varepsilon_t)\right] = 0
  \end{equation}
  with $\varepsilon_t \sim \iid{0}{\Sigma_{\varepsilon}}$ is a random vector ($r \times 1$) of structural innovations,
  $ y_t \in \Lambda\subseteq\mathbb R^n $ a vector of endogenous variables,
  $\mathcal F_{\theta}: \Lambda^{3} \times \mathbb R^r \rightarrow \Lambda$ a
  real function in $\mathcal C^2$ parameterized by a real vector
  $\theta \in \Theta \subseteq \mathbb R^q$ gathering the deep parameters of the model.

  \bigskip

  \item The model is stochastic, forward looking and non linear.

  \bigskip

  \item We want to estimate (a subset of) $\theta$. For any estimation approach
(\textcolor{gray}{indirect inference},
   \textcolor{gray}{simulated moments}, maximum likelihood,...) we need first to solve this model.

  \end{itemize}

\end{frame}

\begin{frame}
  \frametitle{Estimation of DSGE models}
  \framesubtitle{Reduced form}
  \begin{itemize}
    \item In the sequel we consider a local linear approximation
      around the deterministic steady state of the non linear model.
      \bigskip
    \item The solution of the linearized model (the reduced form)
      is:
      \[
      y_t = \bar y(\theta) + A(\theta) \left( y_{t-1} - \bar y(\theta)\right) + B(\theta)\varepsilon_t
      \]
      where the steady state, $\bar y(\theta)$, and matrices
      $A(\theta)$ and $B(\theta)$ are nonlinear functions of the deep
      parameters.
      \bigskip
    \item The unconditional covariance matrix of $y$ solves:
      \[
      \Sigma_y = A(\theta)\Sigma_yA(\theta)' + B(\theta)\Sigma_{\varepsilon}B(\theta)'
      \]
      and the autocovariance function is defined as:
      \[
      \Gamma_h = A(\theta)\Gamma_{h-1}\quad\forall h\geq 1
      \]
      with $\Gamma_0 =\Sigma_y $ and $\Gamma_{-h} = \Gamma_h'$.
  \end{itemize}

\end{frame}


\begin{frame}
  \frametitle{Estimation of DSGE models}
  \framesubtitle{Likelihood}
  \begin{itemize}
    \item Let $\mathcal Y_T =
      {y_1^{\star},y_2^{\star},\dots,y_T^{\star}}$ be the sample, with
      \[
      y_t^{\star} = Z y_t
      \]
      where $Z$ is a $p\times n$ selection matrix.
      \bigskip
    \item The likelihood is the density of the sample. If the
      structural innovations are Gaussian:
      \[
      \mathcal L = (2\pi)^{-\frac{pT}{2}}|\Sigma_{\mathbf
        y^{\star}}|^{-\frac{1}{2}}e^{-\frac{1}{2}(\mathbf y^{\star}-\bar{\mathbf y}^{\star})'\Sigma_{\mathbf
        y^{\star}}^{-1}(\mathbf y^{\star}-\bar{\mathbf y}^{\star})}
      \]
      with $\mathbf y^{\star} = (\left.\mathbf
        y_1^{\star}\right.',\left.\mathbf
        y_2^{\star}\right.',\dots,\left.\mathbf y_T^{\star}\right.')'$
      a $Tp\times 1$ vector, and
      \[
      \Sigma_{\mathbf y^{\star}} =
      \begin{pmatrix}
        \Gamma_0^{\star}  & \Gamma_1^{\star}  & \Gamma_2^{\star} &  \dots   & \dots & \dots & \Gamma_{T-1}^{\star}\\
        {\Gamma_1^{\star}}' & \Gamma_0^{\star}  & \Gamma_1^{\star} & \Gamma_2^{\star} & \dots & \dots & \Gamma_{T-2}^{\star}\\
        {\Gamma_2^{\star}}' & {\Gamma_1^{\star}}' & \Gamma_0^{\star} & \Gamma_1^{\star} & \dots & \dots & \Gamma_{T-3}^{\star}\\
        \vdots    &           &         &           &      &        & \\
        {\Gamma_{T-1}^{\star}}' & {\Gamma_{T-2}^{\star}}' & \dots & \dots & \dots & {\Gamma_1^{\star}}' & \Gamma_{0}^{\star}\\
      \end{pmatrix}
      \]
      where $\Gamma_h^{\star} = Z\Gamma_hZ'$.
  \end{itemize}

\end{frame}


\begin{frame}
  \frametitle{Estimation of DSGE models}
  \framesubtitle{Kalman filter}
  \begin{itemize}
  \item The well known Kalman filter bayesian recursive algorithm can be used to evaluate the likelihood:
  \begin{eqnarray*}
    v_t &=& y^{\star}_t - \bar y(\theta)^{\star} - Z \hat y_t\\
    F_t &=& Z P_t Z' {\color{blue} + \mathbb V \left[\eta\right]}\\
    K_t &=& A(\theta) P_t A(\theta)'F_t^{-1}\\
    \hat y_{t+1} &=& A(\theta) \hat y_t + K_tv_t\\
    P_{t+1} &=& A(\theta) P_t (A(\theta)-K_t Z)'+B(\theta)\Sigma_{\varepsilon} B(\theta)'
  \end{eqnarray*}
  for $t=1,\ldots,T$, with initial condition
  $\hat{y}_0=y_0^{\star}-\bar y^{\star}$ and $P_0$ given by the ergodic
  distribution of $y^{\star}$.

  \bigskip

  \item The (log)-likelihood is:
    \[
      \ln \mathcal L = -\frac{Tp}{2}\ln(2\pi)-\frac{1}{2}\sum_{t=1}^T|F_t|-\frac{1}{2}v_t'F_t^{-1}v_t
    \]

  \item $F_t$ is full rank only if we have at least as many innovations as observed variables.

  \end{itemize}

\end{frame}


\begin{frame}
  \frametitle{Estimation of DSGE models}
  \framesubtitle{Posterior distribution}

  \begin{itemize}

  \item We know that the posterior density
    is proportional to the likelihood times the prior density.

\bigskip

  \item Because we do not have closed form expressions for the reduced
    form of the model and the likelihood, we cannot obtain analytical
    results about the posterior density.

\bigskip

  \item Posterior inference can be done by considering:

    \begin{enumerate}
    \item Asymptotic (Gaussian) approximation of the posterior density.
    \item Simulation based methods (exact up to the randomness
      inherent to these methods).
    \end{enumerate}

\end{itemize}

\end{frame}

\section{Simulation based posterior inference}

\begin{frame}
  \frametitle{Simulation based posterior inference}

  \begin{itemize}

  \item We need a simulation approach if we want to obtain exact
    results  (\textit{ie} not relying on asymptotic approximation).

    \bigskip

  \item Noting that:
    \[
    \mathbb E \left[ \varphi(\psi) \right] = \int_{\Psi} \varphi (\psi) p_1(\psi|\sample)\mathrm d\psi
    \]
    we can use the empirical mean of $\left(\varphi(\psi^{(1)}),\varphi(\psi^{(2)}),\dots,\varphi(\psi^{(n)})\right)$,
    where $\psi^{(i)}$ are draws from the posterior distribution to evaluate the expectation of $\varphi
    (\psi)$. The approxomation error goes to zero when $n\rightarrow\infty$.

\bigskip

  \item We need to simulate draws from the posterior distribution.\newline

    $\Rightarrow$ Metropolis-Hastings algorithm.

\end{itemize}
\end{frame}



\begin{frame}
  \frametitle{Simulation based posterior inference}
  \framesubtitle{A simple example -- a --}

    \begin{itemize}
        \item Imagine we want to obtain some draws from a $\mathcal N (0,4)$
        distribution...

        \bigskip

        \item But we are only able to draw from $\mathcal N (0,1)$
        and we don't realize that we should simply multiply by 2 the
        draws from a standard normal distribution.

        \bigskip

        \item The idea is to build a stochastic process whose limiting
        distribution is $\mathcal N (0,4)$.

        \bigskip

        \item We define the following AR(1) process:
        \[
            x_t = \rho x_{t-1} + \epsilon_t
        \]
        with $\epsilon_t \sim \mathcal N (0,1)$, $|\rho|<1$ and $x_0 =
        0$.

        \bigskip

        \item We just have to choose $\rho$ such that the asymptotic
        distribution of $\{x_t\}$ is $\mathcal N (0,4)$.
    \end{itemize}
\end{frame}


\begin{frame}
  \frametitle{Simulation based posterior inference}
  \framesubtitle{A simple example -- b --}
    We have:
    \begin{itemize}
        \item $x_1 = \epsilon_1 \sim \mathcal N(0,1)$
        \item $x_2 = \rho \epsilon_1 + \epsilon_2  \sim \mathcal N\left(0,1+\rho^2\right)$
        \item $x_3 = \rho^2 \epsilon_1 + \rho\epsilon_2+\epsilon_3  \sim \mathcal
N\left(0,1+\rho^2+\rho^4\right)$
        \item ...
        \item $x_T = \rho^{T-1} \epsilon_1 +
          \rho^{T-2}\epsilon_2+\dots+\epsilon_{T} \sim \mathcal
          N\left(0,1+\rho^2+\dots\rho^{2(T-1)}\right)$
        \item ...
        \item And asymptotically
        \[
            x_{\infty}\sim \mathcal N\left(0,\frac{1}{1-\rho^2}\right)
        \]
        So that $\mathbb V_{\infty}[x_t] = 4$ iff $\rho = \pm
        \frac{\sqrt{3}}{2}$.
    \end{itemize}
\end{frame}


\begin{frame}
  \frametitle{Simulation based posterior inference}
  \framesubtitle{A simple example -- c --}

    \begin{itemize}
        \item If we simulate enough draws from this Gaussian
        autoregressive stochastic process, we can replicate the
        targeted distribution.

\bigskip

        \item In this case it is very simple because we know exactly
        the targeted distribution \textbf{and} we are able to obtain
        some draws from its standardized version.

\bigskip

        \item This is far from true with \textsc{dsge} models. For
        instance, we even don't have an analytical expression for
        the posterior density.

    \end{itemize}

\end{frame}


\begin{frame}
  \frametitle{Simulation based posterior inference}
  \framesubtitle{Metropolis--Hastings algorithm}

    \begin{enumerate}
    \item Choose a starting point $\Psi^0$ (usually the posterior
      mode) and run a loop over 2-3-4.

\bigskip

        \item Draw a \emph{proposal} $\Psi^{\star}$ from a \emph{jumping} distribution
        \[
            J(\Psi^{\star}|\Psi^{t-1}) =
            \mathcal N(\Psi^{t-1},c\times\Omega_{m})
        \]

\bigskip

        \item Compute the acceptance ratio
        \[
            r = \frac{p_1(\Psi^{\star}|\sample)}{p_1(\Psi^{t-1}|\sample)} = \frac{\mathcal
            K(\Psi^{\star}|\sample)}{\mathcal K(\Psi^{t-1}|\sample)}
        \]

\bigskip

        \item Finally
        \[
            \Psi^t = \left\{
            \begin{array}{ll}
                \Psi^{\star} & \mbox{ with probability $\min(r,1)$}\\
                \Psi^{t-1} & \mbox{ otherwise.}
            \end{array}\right.
        \]
    \end{enumerate}
\end{frame}



\begin{frame}
  \frametitle{Simulation based posterior inference}
  \framesubtitle{Metropolis--Hastings algorithm illustration -- a --}
  \bigskip
  \begin{center}
  \scalebox{.5}{\input{../img/mhplot-a.tex}}
  \end{center}
\end{frame}

\begin{frame}
  \frametitle{Simulation based posterior inference}
  \framesubtitle{Metropolis--Hastings algorithm illustration -- b --}
  \bigskip
  \begin{center}
  \scalebox{.5}{\input{../img/mhplot-b.tex}}
  \end{center}
\end{frame}


\begin{frame}
  \frametitle{Simulation based posterior inference}
  \framesubtitle{Metropolis--Hastings algorithm illustration -- c --}
  \bigskip
  \begin{center}
  \scalebox{.5}{\input{../img/mhplot-c.tex}}
  \end{center}
\end{frame}


\begin{frame}
  \frametitle{Simulation based posterior inference}
  \framesubtitle{Tuning of the Metropolis--Hastings algorithm}
  \begin{itemize}
  \item How should we choose the scale factor $c$ (variance of the jumping
    distribution)?
  \item We define the acceptance ratio as the number of accepted
    draws over the number of proposals:
    \[
    R = \frac{\# \text{ accepted draws}}{\# \text{ proposals}}
    \]
    {\footnotesize
      \begin{itemize}
      \item We do not want to have $R$ close to 0 because it means
        that we reject almost all the proposals.
      \item We neither want to have $R$ close to 1 because it means
        that we accept almost all the proposals and that most
        likely the jumping distribution proposes too small jumps.
      \end{itemize}}
  \item In the literature, authors generally target an acceptance
    ratio around one third, although there is no rational for
    that choice in the case of DSGE models.
  \item The scale factor, $c$, must be adjusted to match this
    target:
    {\footnotesize
      \begin{itemize}
      \item If $R$ is too low, $c$ should be reduced (why?).
      \item If $R$ is too high, $c$ should be increased (why?).
      \end{itemize}}
  \end{itemize}
\end{frame}


\begin{frame}
  \frametitle{Simulation based posterior inference}
  \framesubtitle{Convergence of the Metropolis--Hastings algorithm}
  \begin{itemize}
  \item Another issue is the number of draws required to obtain
    an accurate estimation of the posterior distribution.
  \item We need to asses the convergence of the Metropolis
    Hastings algorithm.
  \item The estimated posterior distribution should be:
    {\footnotesize
      \begin{enumerate}[(i)]
      \item stable when we increase the number of simulations.
      \item independent of the initial condition.
      \end{enumerate}}
  \item To check that the estimated posterior distribution
    satisfy these two properties, we can run multiple
    Monte Carlo Markov Chains  and verify that:
    {\footnotesize
      \begin{enumerate}[(i)]
      \item Moments are constant if the number of simulations is increased.
      \item Pooled and Within moments are identical.
      \end{enumerate}}
  \item This approach is implemented in Dynare (see the manual).
  \end{itemize}
\end{frame}


\section{Marginal density estimation}

\begin{frame}
  \frametitle{Marginal density estimation}
  \begin{itemize}
  \item The marginal density of the sample may be written as:
    \[
    p(\sample|\mathcal{A}) = \int_{\Psi_{\mathcal{A}}}
    p(\sample,\psi_{\mathcal{A}}|\mathcal{A})\mathrm d\psi_{\mathcal{A}}
    \]
    \bigskip
  \item ... or equivalently:
    \[
    p(\sample|\mathcal{A}) = \int_{\Psi_{\mathcal{A}}}
    \underbrace{p(\sample|\Psi_{\mathcal{A}},\mathcal{A})}_{\text{likelihood}}
    \underbrace{p_0(\psi_{\mathcal{A}}|\mathcal{A})}_{\text{prior}}\mathrm d\psi_{\mathcal{A}}
    \]
    \bigskip
  \item We face an integration problem.
  \end{itemize}

\end{frame}


\begin{frame}
  \frametitle{Marginal density estimation}
  \framesubtitle{Asymptotic approximation}

  \begin{itemize}

  \item For \textsc{dsge} models we are unable to compute this
    integral analytically or with standard numerical tools (because of
    the curse of dimensionality).

\bigskip

  \item We assume that the posterior distribution is not too far from
    a gaussian distribution. In this case we can approximate the
    marginal density of the sample.

\bigskip

  \item We have (omitting the conditioning on the model):
    \medskip
    \[
    p(\sample) \approx
    (2\pi)^{\frac{n}{2}}|\mathcal
    H(\psi^{\ast})|^{\frac{1}{2}}p(\sample|\psi^{\ast})
    p_0(\psi^{\ast})
    \]

\bigskip

  \item This approach gives accurate estimation of the marginal
  density if the posterior distribution is uni-modal.

\end{itemize}

\end{frame}


\begin{frame}
  \frametitle{Marginal density estimation}
  \framesubtitle{A first simulation based method}

  \begin{itemize}

  \item We can estimate the marginal density using a Monte-Carlo approach
    \[
    \widehat{p}(\sample) = \frac{1}{B}\sum_{b=1}^B
    p(\sample|\psi^{(b)})
    \]
    where $\psi^{(b)}$ is sampled from the prior distribution.

\bigskip

  \item $\widehat{p}(\sample)
    \underset{B\rightarrow\infty}{\longrightarrow}
    p(\sample)$.

\bigskip

  \item But this method is highly inefficient,
    because:
    \begin{itemize}
    \item $\widehat{p}(\sample)$
      may have a huge variance (even infinite in some pathological cases).
    \item We are not using simulations already done
      to obtain the posterior distribution (\textit{ie}
      Metropolis-Hastings draws).
    \end{itemize}
  \end{itemize}

\end{frame}


\begin{frame}
  \frametitle{Marginal density estimation}
  \framesubtitle{Harmonic mean -- a --}

    \begin{itemize}
        \item For any probability density function $f$, we have:
        {\small
            \[
            \mathbb{E}\left[\frac{f(\psi)}{p_0(\psi)
                p(\sample|\psi)}\biggl|\psi\biggr.\right] =
            \int_{\Psi}
            \frac{f(\psi)p_1(\psi|\sample)}{p_0(\psi) p(\sample|\psi)}
            \mathrm d\psi
            \]
        }

\bigskip

\item Using the definition of the posterior density:
  {\footnotesize
            \[
                \int_{\Psi}
                \frac{f(\psi)}{p_0(\psi)
                p(\sample|\psi)}\frac{p_0(\psi)p(\sample|\psi)}
                {\int_{\Psi}p_0(\psi)p(\sample|\psi)\mathrm d\psi}\mathrm d\psi
            \]
        }

\bigskip

        \item Finally
        {\footnotesize
            \[
                \mathbb{E}\left[\frac{f(\psi)}{p_0(\psi)
                p(\sample|\psi)}\biggl|\psi\biggr.\right]
                =
                \frac{\int_{\Psi}f(\psi)\mathrm d\psi}{\int_{\Psi}p_0(\psi)
                p(\sample|\psi)\mathrm d\psi}
            \]
        }

    \end{itemize}

\end{frame}


\begin{frame}
  \frametitle{Marginal density estimation}
  \framesubtitle{Harmonic mean -- b --}


    \begin{itemize}
        \item So that
        \[
            p(\sample)=\mathbb{E}\left[\frac{f(\psi)}{p_0(\psi)p(\sample|\psi)}\biggl|\psi\biggr.\right]^{-1}
        \]

\bigskip

        \item This suggests the following estimator of the marginal
        density
        \[
            \widehat{p}(\sample)= \left[\frac{1}{B}\sum_{b=1}^B
            \frac{f(\psi^{(b)})}{p_0(\psi^{(b)})
            p(\sample|\psi^{(b)})}\right]^{-1}
        \]

\bigskip

        \item Each drawn vector $\psi^{(b)}$ comes from the
        Metropolis - Hastings monte-carlo simulations.

\end{itemize}

\end{frame}


\begin{frame}
  \frametitle{Marginal density estimation}
  \framesubtitle{Harmonic mean -- c --}

    \begin{itemize}
        \item The preceding proof holds if we replace
        $f(\theta)$ by 1\\ $\looparrowright$ Simple Harmonic Mean
        estimator. But this estimator may also have a huge
        variance.

\bigskip

        \item The density $f(\theta)$ may be interpreted as a
        weighting function, we want to give less importance to extremal values of
        $\theta$.

\bigskip

        \item Geweke (1999) suggests to use a truncated gaussian
        function (modified harmonic mean estimator).

    \end{itemize}

\end{frame}

\begin{frame}
  \frametitle{Marginal density estimation}
  \framesubtitle{Harmonic mean -- d --}


{\footnotesize
\[ \overline{\psi} =
\frac{1}{B}\sum_{b=1}^B\psi^{(b)}
\]
}
{\footnotesize
\[ \overline{\Omega} =
\frac{1}{B}\sum_{b=1}^B(\psi^{(b)}-\overline{\psi})'(\psi^{(b)}
-\overline{\psi})
\]
}

\bigskip

\begin{itemize}
    \item For some $p\in (0,1)$ we define
    {\footnotesize
    \[\widetilde{\Psi} =
\left\{\psi:(\psi^{(b)}-\overline{\psi})'\overline{\Omega
}^{-1}(\psi^{(b)}-\overline{\psi})\leq
    \chi_{1-p}^2(n)\right\}
    \]}

\bigskip

    \item ... And take
    {\footnotesize
    \[
        f(\psi) =
p^{-1}(2\pi)^{-\frac{n}{2}}|\overline{\Omega}|^{-\frac{1}{2}}e^{-\frac{1}{2}(\psi
-\overline{\psi})'\overline{\Omega}^{-1}(\psi-\overline{\psi})}\mathbb{I}_{
\widetilde{\Psi}}(\psi)
    \]}
\end{itemize}

\end{frame}


\section{Posterior inference}

\begin{frame}
  \frametitle{Posterior inference}
  \framesubtitle{Credible set}

    \begin{itemize}
    \item A synthetic way to characterize the posterior distribution
    is to build something like a confidence interval.

\bigskip

    \item We define:
    \[
        P(\psi \in C) = \int_C p(\psi)\mathrm d\psi = 1-\alpha
    \]
    is a $100(1-\alpha)\%$ credible set for $\psi$ with respect to
    $p(\psi)$ (for instance, with $\alpha=0.2$ we have a 80\% credible set).

\bigskip

    \item A $100(1-\alpha)\%$ highest probability density (HPD) credible set
    for $\psi$ with respect to $p(\psi)$ is a $100(1-\alpha)\%$
    credible set with the property
    \[
        p(\psi_1) \ge p(\psi_2)\;\;\; \forall \psi_1 \in C \mbox{ and
        } \forall \psi_2 \in \bar C
    \]
    \end{itemize}

  \end{frame}


\begin{frame}
  \frametitle{Posterior inference}
  \framesubtitle{Density}

    \begin{itemize}

    \item To obtain a view of the posterior distribution we can
    estimate the marginal posterior densities (for each parameter of the model).

\bigskip

    \item We use a non parametric estimator:
    {
    \[
        \hat{f}(\psi) = \frac{1}{Nh}\sum_{i=1}^N K\left(\frac{\psi-\psi^{(i)}}{h}\right)
    \]
    }
    where $N$ is the number of draws in the metropolis, $\psi$ is
    a  point where we want to evaluate the posterior density,
    $\psi^{(i)}$ is a draw from the metropolis, $K(\bullet)$ is a
    kernel (gaussian by default in Dynare) and $h$ is a bandwidth
    parameter.

\bigskip

    \item In Dynare the bandwidth parameter is, by default, optimally
    chosen considering the Silverman's rule of thumb (controling for
    the repetitions in the MCMC draws).
    \end{itemize}

\end{frame}


\begin{frame}
  \frametitle{Posterior inference}
  \framesubtitle{Predictive density}

    \begin{itemize}

    \item Knowing the posterior distribution of the model's
    parameters, we can forecast the endogenous variables of the
    model.

\bigskip

    \item We define the posterior predictive density as follows:
    \[
p({\tilde{\mathbf Y}}|\sample) = \int_{\Psi}
p(\tilde{\mathbf Y}, \psi|\sample)\mathrm d\psi
    \]
    where, for instance, $\tilde{\mathbf Y}$ might be $y_{T+1}$.
    Knowing that $p(\tilde{\mathbf Y}, \psi|\sample) = p(\tilde{\mathbf Y}|
\psi,\sample)p_1(\psi|\sample)$
    we have:
  \[
    p({\tilde{\mathbf Y}}|\sample)
    = \int_{\Psi} p(\tilde{\mathbf Y}| \psi,\sample)p_1(\psi|\sample)\mathrm d\psi
  \]

  \end{itemize}
\end{frame}


\begin{frame}
  \frametitle{Posterior inference}
  \framesubtitle{Predictive density}

    \begin{itemize}

  \item In practice, we just have to
    {\small
    \begin{enumerate}
    \item sample vectors of parameters from
      the posterior distribution (using the MCMC draws),
    \item compute the forecast (also IRFs if needed) for each
      vector of parameters.
    \end{enumerate}}

\bigskip

   \item In the end of this process we obtain an empirical posterior
     distribution for the forecasts (IRFs).

\bigskip

   \item If our loss function is quadratic, we can then report a point
     forecast by computing the mean of this empirical posterior distribution.
\end{itemize}

\end{frame}


\begin{frame}
  \frametitle{Posterior inference}
  \framesubtitle{Integration}

  \begin{itemize}

    \item More generally, the MCMC draws can be used to estimate any moments
    of the parameters (or function of the parameters).

    \bigskip

    \item We have
    \begin{eqnarray*}
        \mathbb E\left[h(\psi)\right] &=&
\int_{\Psi}h(\psi)p(\psi|\sample)\mathrm d\psi\\
        &\approx& \frac{1}{N}\sum_{i=1}^Nh\left(\psi^{(i)}\right)
    \end{eqnarray*}
    where $\psi^{(i)}$ is a metropolis draw and $h$ is any continuous function.
  \end{itemize}
\end{frame}


\begin{frame}[fragile]
  \frametitle{Estimation in pratice}

  \begin{itemize}

  \item Declare the set of observed variables with \verb+varobs+.

  \item Declare the priors with the \verb+estimated_params+ block (available priors: uniform, gamma, inverse gamma, gaussian, beta and weibull).

  \item Use the estimation command

    \begin{itemize}
    \item Estimates the posterior mode (used as an initial condition for MCMC),
    \item Runs the MCMC,
    \item Computes posterior moments.
    \end{itemize}

  \item Possible trouble with the first step if the hessian matrix is not positive $\Rightarrow$ The estimated posterior mode is not a mode. $\Rightarrow$ Try other optimization algorithms and/or different initial conditions.

  \end{itemize}
\end{frame}



\end{document}




% Local Variables:
% ispell-check-comments: exclusive
% ispell-local-dictionary: "american-insane"
% End:
