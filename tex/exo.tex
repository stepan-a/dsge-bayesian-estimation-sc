\documentclass[a4paper,10pt]{amsart}

\usepackage[a4paper]{geometry}
\usepackage{amssymb}
\usepackage{float}
\usepackage{setspace}
\usepackage{mathrsfs}
\usepackage{graphicx}
\usepackage{psfrag}
\usepackage{dcolumn}
\usepackage{multirow}
\usepackage{hyperref}
\usepackage{subfigure}
\usepackage{algorithms/algorithm}
\usepackage{algorithms/algorithmic}
\usepackage[english]{babel}
\usepackage[round]{natbib}
\usepackage[nice]{nicefrac}
\usepackage{fancybox}
\usepackage{multicol}
\usepackage{pgf}
\usepackage{tikz}

\usepackage[utf8]{inputenc}
\usepackage[T1]{fontenc}
\usepackage{hyperref}

\newcolumntype{.}{D{.}{.}{-1}}
\newcolumntype{a}{D{.}{.}{10}}
\setlength{\parindent}{0pt}

\pdfcompresslevel=9

\hypersetup{
    bookmarks=false,
    unicode=true,
    pdftoolbar=false,
    pdfmenubar=false,
    pdffitwindow=true,
    pdfstartview={FitH},
    pdftitle={Estimation of an RBC model with Dynare},
    pdfauthor={Stéphane Adjemian},
    pdfcreator={Stéphane Adjemian},
    pdfproducer={pdflatex},
    colorlinks=true,
    linkcolor=blue,
    citecolor=blue,
    filecolor=blue,
    urlcolor=blue
}

\begin{document}

\title{Estimation of an RBC model with Dynare}
\author{Stéphane Adjemian}
\date{}

\maketitle

Consider the following RBC model (social planner program):
\begin{equation}
  \label{eq:social-planner-problem:1}\tag{SP$_{1}$-1}
  \begin{split}
    \max_{\{c_{t+j},l_{t+j},k_{t+1+j}\}_{j=0}^{\infty}} &\quad \mathcal W_t = \sum_{j=0}^{\infty}\beta^ju(c_{t+j},l_{t+j})\\
           \underline{s.t.}   &  \\
              \qquad y_t &= c_t + i_t\\
              \qquad y_t &= A_tf(k_t,l_t)\\
              \qquad k_{t+1} &= i_t + (1-\delta)k_t\\
              \qquad A_{t} &= {A^{\star}}e^{a_{t}}\\
              \qquad a_{t} &= \rho a_{t-1} + \varepsilon_t
  \end{split}
\end{equation}
where preferences and technology are characterized as follows:
\begin{equation}
  \label{eq:utility}
  u(c_t,l_t) = \frac{\left(c_t^{\theta}(1-l_t)^{1-\theta}\right)^{\tau}}{1-\tau}
\end{equation}
and
\begin{equation}
  \label{eq:production}
  f(k_t,l_t) = \left(\alpha k_t^{\psi} + (1-\alpha)l_t^{\psi}\right)^{\frac{1}{\psi}}
\end{equation}
and $\varepsilon_t$ is a Gaussian white noise with zero mean and variance $\sigma_{\varepsilon}^2$. The first order conditions are given by:
\begin{subequations}
  \label{eq:foc:1}
  \begin{equation}
    \label{eq:foc:1:euler}
    u_c(c_t,l_t) - \beta \mathbb E_t\left[ u_c(c_{t+1},l_{t+1})\Bigl(A_{t+1}f_k(k_{t+1},l_{t+1}) + 1 -\delta\Bigr) \right] = 0
  \end{equation}
  \begin{equation}
    \label{eq:foc:1:labour}
    -\frac{u_{l}(c_t,l_t)}{u_c(c_t,l_t)} - A_tf_l(k_t,l_t) = 0
  \end{equation}
  \begin{equation}
    \label{eq:cno:1:capital_law_of_motion}
    c_t + k_{t+1} - A_tf(k_t,l_t)- (1-\delta)k_t = 0
  \end{equation}
\end{subequations}

\bigskip

\textbf{(1)}  Write  a  mod  file  for this  model  (with  a  sensible
calibration of the parameters and a steady state block)\footnote{An
  example is available \href{https://github.com/stepan-a/dsge-bayesian-estimation-sc/tree/master/mod}{here}.}.\newline

\textbf{(2)} Using the \verb+stoch_simul+ command simulate a sample of
10000 observations and save it in a file.\newline

\textbf{(3)} Define priors over a subset of the parameters you want to
estimate.\newline

\textbf{(4)} Estimate  the posterior mode  (with the \verb+estimation+
command)\footnote{If  you  consider  that  more  than  one  endogenous
  variable is observed you will  run into problems.  Why? How to solve
  this issue?},  considering a sample of 100  observations.  Check the
estimated  posterior mode (using  the \verb+mode_check+  command).  If
Dynare  warns you  saying  that  the hessian  matrix  is not  positive
definite,  use  another  optimization  algorithm  starting  from  your
previous  estimate  of  the  posterior  mode  or  change  the  initial
conditions.\newline

\textbf{(5)}  Once   you  are   satisfied  with  the   posterior  mode
estimation,  run  a  MCMC  with $3\times5000$  iterations.  Check  the
convergence  diagnostics.  If  the  MCMC  did  not  converge  to  the
(ergodic) posterior distribution rerun a metropolis without discarding
the previous draws.\newline


\textbf{(6)} Evaluate  the robustness of your results  with respect to
the  specification of the  prior beliefs  (re-estimate the  prior mode
with different priors).\newline

\textbf{(7)} Using  the  same   dataset,  estimate   a  misspecified
model\footnote{Obviously, you  also need to adapt  the steady state.}
(for instance the same  model with a Cobb-Douglas production function,
or the same model with a separable utility function, or the same model
with  a representative  household offering  inelastically one  unit of
labour). Compare the estimates  of the common deep parameters. Compare
the marginal densities of the different models.

\end{document}